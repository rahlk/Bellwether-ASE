\documentclass{sig-alternate}
%\usepackage{stfloats}
\usepackage{tikz}
  \def\firstcircle{(90:1.75cm) circle (2.5cm)}
  \def\secondcircle{(210:1.75cm) circle (2.5cm)}
  \def\thirdcircle{(330:1.75cm) circle (2.5cm)} 
\usepackage{comment}
\usepackage{cite}
\usepackage{framed,graphicx,xcolor}
\usepackage{multirow}
\usepackage{rotating}
%\usepackage{stfloats}

\usepackage[shortlabels]{enumitem} 
\usepackage{amsmath}
\usepackage{ amssymb }
\usepackage{url}
\usepackage{balance}
\newcommand{\bi}{\begin{itemize}}
\newcommand{\ei}{\end{itemize}}
\newcommand{\be}{\begin{enumerate}}
\newcommand{\ee}{\end{enumerate}}
\newcommand{\tion}[1]{\textsection\ref{sect:#1}}
\newcommand{\fig}[1]{Figure~\ref{fig:#1}}
\newcommand{\eq}[1]{Equation~\ref{eq:#1}}
%\setlist{nolistsep,leftmargin=5mm}
%\usepackage[pdftex]{graphicx}
\usepackage{program}
\newcommand{\Sample}{{\bf SAMPLE}}
\newcommand{\PEEKING}{{\bf PEEKING2}}
%\usepackage[table]{xcolor}
\definecolor{darkgreen}{rgb}{0,0.3,0}
\definecolor{Gray}{rgb}{0.88,1,1}
\definecolor{Gray}{gray}{0.85}
\definecolor{Blue}{RGB}{0,29,193}
\usepackage{colortbl}
\usepackage{picture}
\usepackage{url}
\usepackage{hyperref}
%\usepackage{listings}
\DeclareMathOperator*{\argmin}{arg\,min} 
\DeclareMathOperator*{\argmax}{arg\,max}
\definecolor{lightgray}{gray}{0.8}
\definecolor{darkgray}{gray}{0.6}
\definecolor{Gray}{gray}{0.95}
\definecolor{LightGray}{gray}{0.975}

\definecolor{Code}{rgb}{0,0,0}
\definecolor{Decorators}{rgb}{0.5,0.5,0.5}
\definecolor{Numbers}{rgb}{0.5,0,0}
\definecolor{MatchingBrackets}{rgb}{0.25,0.5,0.5}
\definecolor{Keywords}{rgb}{0,0,1}
\definecolor{self}{rgb}{0,0,0}
\definecolor{Strings}{rgb}{0,0.63,0}
\definecolor{Comments}{rgb}{0,0.63,1}
\definecolor{Comments}{rgb}{0.5,0.5,0.5}
\definecolor{Backquotes}{rgb}{0,0,0}
\definecolor{Classname}{rgb}{0,0,0}
\definecolor{FunctionName}{rgb}{0,0,0}
\definecolor{Operators}{rgb}{0,0,0}
\definecolor{Background}{rgb}{1,1,1}
\bibliographystyle{unsrt}
 \definecolor{lavenderpink}{rgb}{0.98, 0.68, 0.82}
 \definecolor{celadon}{rgb}{0.67, 0.88, 0.69}
\newcommand{\G}{\cellcolor{green}}
\newcommand{\Y}{\cellcolor{yellow}}

\newcommand{\quart}[4]{\begin{picture}(80,4)%1
{\color{black}\put(#3,2){\circle*{4}}\put(#1,2){\line(1,0){#2}}}\end{picture}}


\definecolor{MyDarkBlue}{rgb}{0,0.08,0.45} 
\newenvironment{changed}{\par\color{MyDarkBlue}}{\par}
%\newenvironment{changed}{\par}{\par}

\newcommand{\ADD}[1]{\textcolor{MyDarkBlue}{{\bf #1}}}
\usepackage{times}
\pagenumbering{arabic} 

% \usepackage[svgnames]{xcolor}
   \usepackage[framed]{ntheorem}
\usepackage{framed}
\usepackage{tikz}
\usetikzlibrary{shadows}
%\newtheorem{Lesson}{Lesson}
\theoremclass{Lesson}
\theoremstyle{break}

% inner sep=10pt,
\tikzstyle{thmbox} = [rectangle, rounded corners, draw=black,
  fill=Gray!20,  drop shadow={fill=black, opacity=1}]
\newcommand\thmbox[1]{%
  \noindent\begin{tikzpicture}%
  \node [thmbox] (box){%
\begin{minipage}{.94\textwidth}%    
      \vspace{-3mm}#1\vspace{-3mm}%
    \end{minipage}%
  };%
  \end{tikzpicture}}

\let\theoremframecommand\thmbox
\newshadedtheorem{lesson}{Research answer}
 
 
\begin{document}  

\definecolor{shadecolor}{gray}{0.9}
\conferenceinfo{ASE}{'16 Singapore, Singapore.}

\title{  Exploring ``Bellwether'' Effect in Software Engineering}
\numberofauthors{2} 
\author{  
\alignauthor
Rahul~Krishna, Tim Menzies  \\
       \affaddr{Computer Science, NC State, USA}\\
       {\{i.m.ralk,~tim.menzies\}@gmail.com}
\alignauthor
Lucas Layman \\
       \affaddr{Fraunhofer CESE, College Park, USA}\\ 
       {llayman@cese.fraunhofer.org}
\setlength{\columnsep}{7mm}}
\maketitle

\begin{abstract} 

{\bf Categories/Subject Descriptors:} D.2 [Software Engineering]; I.2 [Artificial Intelligence];

 
{\bf Keywords:} 
\end{abstract}

\section{Introduction}
Numerous {\em local learning} results show that we
should mistrust general conclusions (made over a
wide population of projects) since they may not hold
for projects.  Posnett et al.~\cite{posnet11}
discuss {\em ecological inference} in software
engineering, which is the concept that what holds
for the entire population also holds for each
individual.  They learn models at different levels
of aggregation (modules, packages, files) and show
that models that work at one level of aggregation
can be sub-optimal at others.  For example, Yang et
al.~\cite{yang11}, Bettenburg et
al.~\cite{Bettenburg2012}, and Menzies et al.~\cite{me12d}
all explore the generation of models using {\em all}
data versus {\em local} samples that more specific
to particular test cases. These papers report that
better models (sometimes with much lower variance in
their predictions) are generated from local
information.
These results have an unsettling effect on anyone
struggling to propose policies for an organization.
If all prior conclusions can change for the new
project, or some small part of a project, how can
any manager ever hope to propose and defend IT
policies (e.g., when should some module be inspected,
when should it be refactored, where to focus
expensive testing procedures, etc.)?

If we cannot {\em generalize} to all projects and all parts
of current projects, perhaps a more achievable goal is to {\em
stabilize} the pace of conclusion change.
While it may be
a fool's errand  and wait for  eternal and global SE
conclusions, one possible approach is for organizations
to declare $N$ prior projects as {\em reference projects},
from which lessons learned will be transferred to new projects.
In practice, using such reference sets requires three processes:

\begin{enumerate}
\item Finding the reference sets (this paper shows that finding
 them may not be a too complex task, at least for defect prediction).
 \item Recognizing when to update  the reference set. In practice,
 this could be as simple as noting when predictions start failing for
new projects---at which time, we would loop to point \#1.
\item Transferring
 lessons from the reference set to new projects.
\end{enumerate}

In this approach, the policies of the organization will be
stable just as long as the reference set is not updated.
In this paper, we do not address the pace of change in the reference set
(that is left for future work).
Rather, we focus on point \#3: transferring lessons from
the reference set to new projects. To support this third point,
we need to resolve the problems
 that this paper addresses (data expressed in different terminology
 cannot transfer till there is enough data to match old projects to new).
 
\begin{figure*}[t!]
{\scriptsize 
\centering
\caption{Bellwether Effect}
\label{res:jur}
\begin{tabular}{l|c|l|l|l|c|c|l|l|l|c}
         & Ant & Camel & Ivy & Jedit & Log4j & Lucene & Poi & Velocity & Xalan & Xerces                 \\ \hline
Ant      &     &       &     &       &       & $\times$     &     &          &       & \multicolumn{1}{c}{}  \\ \hline
Camel    &     &       &     &       &       & $\times$     &     &          &       & \multicolumn{1}{c}{}  \\ \hline
Ivy      &     &       &     &       &       & $\times$     &     &          &       & \multicolumn{1}{c}{}  \\ \hline
Poi      &     &       &     &       &       & $\times$     &     &          &       & \multicolumn{1}{c}{}  \\ \hline
Velocity &     &       &     &       &       & $\times$     &     &          &       & \multicolumn{1}{c}{}  \\ \hline
Xalan    &     &       &     &       &       & $\times$     &     &          &       & \multicolumn{1}{c}{}  \\ \hline
Xerces   &     &       &     &       &       & $\times$     &     &          &       & \multicolumn{1}{c}{}  \\ \hline
Jedit    & $\times$  &       &     &       &       &        &     &          &       & \multicolumn{1}{c}{}  \\ \hline
Lucene   &     &       &     &       & $\times$    &        &     &          &       & \multicolumn{1}{c}{}  \\ \hline
Log4j    &     &       &     &       &       &        &     &          &       & \multicolumn{1}{c}{$\times$}
\end{tabular}}
\end{figure*}
\balance
\bibliographystyle{unsrt}
\bibliography{References}
\end{document}
